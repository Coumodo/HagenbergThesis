\documentclass[
	hyperref={unicode},	%% must match the hyperref options used by hgb.sty (overrule with hypersetup below)
	url={hyphens},
	english
	]{beamer}
%
% Choose how your presentation looks.
%
% For more themes, color themes and font themes, see:
% http://deic.uab.es/~iblanes/beamer_gallery/index_by_theme.html
%
\mode<presentation>
{
  \usetheme{default}      % or try Darmstadt, Madrid, Warsaw, ...
  \usecolortheme{default} % or try albatross, beaver, crane, ...
  %\usefonttheme{default}  % or try serif, structurebold, ...
	%\usefonttheme[onlymath]{serif}
	\usefonttheme{professionalfonts}
  \setbeamertemplate{navigation symbols}{}
  \setbeamertemplate{caption}[numbered]
} 

\usepackage[utf8]{inputenc}

\usepackage{hgb}
\RequirePackage{hgbmath}
\RequirePackage{hgbabbrev}
\RequirePackage{hgblistings}

\hypersetup{
	colorlinks = true,	% use colored links (no boxes)	
}

%%https://tex.stackexchange.com/questions/24371/does-enumitem-conflict-with-beamer-for-lists/24491#24491
% This handles the incopatibility between beamer and enumitem (used by hgb.sty):
\setitemize{label=\usebeamerfont*{itemize item}%
  \usebeamercolor[fg]{itemize item}
  \usebeamertemplate{itemize item}}

\title[Your Short Title]{Title of Your Presentation}
\author{Your Name}
\institute{Where You're From}
\date{\today}

\begin{document}

\begin{frame}
  \titlepage
\end{frame}

% Uncomment these lines for an automatically generated outline.
%\begin{frame}{Outline}
%  \tableofcontents
%\end{frame}

\section{Introduction}

\begin{frame}{Introduction}

\begin{itemize}
  \item Your introduction goes here!
  \item Use \texttt{itemize} to organize your main points.
\end{itemize}

\vskip 1cm

\begin{block}{Examples}
Some examples of commonly used commands and features are included, to help you get started.
\end{block}

\end{frame}

\section{Some \LaTeX{} Examples}


\subsection{Tables and Figures}

\begin{frame}{Tables and Figures}

\begin{itemize}
\item Use \texttt{tabular} for basic tables --- see Table~\ref{tab:widgets}, for example.
\item You can upload a figure (JPEG, PNG or PDF) using the files menu. 
\item To include it in your document, use the \texttt{includegraphics} command (see the comment below in the source code).
\end{itemize}

% Commands to include a figure:
%\begin{figure}
%\includegraphics[width=\textwidth]{your-figure's-file-name}
%\caption{\label{fig:your-figure}Caption goes here.}
%\end{figure}

\begin{table}
\centering
\begin{tabular}{l|r}
Item & Quantity \\\hline
Widgets & 42 \\
Gadgets & 13
\end{tabular}
\caption{\label{tab:widgets}An example table.}
\end{table}

\end{frame}

\subsection{Mathematics}

\begin{frame}{Readable Mathematics}

Let $X_0, X_1, \ldots, X_{n-1}$ be a sequence of $n$ independent and identically distributed random variables, with $\mathrm{E}[X_i] = \mu$ and $\mathrm{Var}[X_i] = \sigma^2 < \infty$, and let
\begin{equation}
S_n = \frac{X_0 + X_1 + \cdots + X_n}{n-1}
      = \frac{1}{n} \cdot\sum_{i=0}^{n-1} X_i
\label{eq:Mean}
\end{equation}
denote their \emph{mean}. Then as $n$ approaches infinity, the random variables $\sqrt{n} \cdot(S_n - \mu)$ converge 
to a normal distribution $\mathcal{N}(0, \sigma^2)$.

\end{frame}

\end{document}
