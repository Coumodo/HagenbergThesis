\documentclass[english]{hgbarticle}

\RequirePackage[utf8]{inputenc}		% remove when using lualatex oder xelatex!


\title{The \textsf{hagenberg-thesis} Package}
\date{\hgbDate}

\author{W.\ Burger and W.\ Hochleitner\\[10pt]
University of Applied Sciences Upper Austria\\
Department of Digital Media, Hagenberg (Austria)}

%\author{
%Wilhelm Burger, Wolfgang Hochleitner\\ 
%\texttt{wilhelm.burger@fh-ooe.at}
%\and
%Wolfgang Hochleitner\\
%\texttt{wolfgang.hochleitner@fh-ooe.at}}


%%%----------------------------------------------------------
\begin{document}
%%%----------------------------------------------------------
\maketitle
%%%----------------------------------------------------------

\begin{abstract}\noindent
The \textsf{hagenberg-thesis} package is a collection of modern LaTeX templates for university theses (bachelor, 
master or diploma programs) and related documents.
This manual describes the main features of this package.
Pre-configured document templates for English and German manuscripts and a complete
tutorial are available on the package's home repository.
\end{abstract}



\section{Introduction}

The complete source of this package and auxiliary materials are available on
CTAN\footnote{\url{https://ctan.org/pkg/hagenberg-thesis}}
and its development repository.\footnote{\url{https://github.com/Digital-Media/HagenbergThesis}}
The package is made available under the terms of the
Creative Commons Attribution 4.0 International Public License.%
\footnote{\url{https://creativecommons.org/licenses/by/4.0/legalcode}}


\section{Document classes}

The \texttt{hgb} package provides the following document classes,
which are based on the standard \latex\ classes
\texttt{book}, \texttt{report} and \texttt{article}, respectively:
%
\begin{itemize}
\item \textbf{\texttt{hgbthesis}} (\texttt{book}): 
for Bachelor, Master and Diploma theses;
\item \textbf{\texttt{hgbreport}} (\texttt{report}):
for project and term reports;
\item \textbf{\texttt{hgbarticle}} (\texttt{article}):
for drafting journal articles.
\end{itemize}



\subsection{Class options} 
The above document classes accept the following options:
%
\begin{raggedright}
\begin{itemize}
\item \textbf{\texttt{hgbthesis}}: 
	\texttt{master}, \texttt{diploma}, \texttt{bachelor}, \texttt{praktikum}, 
	\texttt{internship}, \texttt{english}, \texttt{german};
\item \textbf{\texttt{hgbreport}}: \texttt{notitlepage}, \texttt{english}, \texttt{german};
\item \textbf{\texttt{hgbarticle}}: \texttt{twocolumn}, \texttt{english}, \texttt{german}.
\end{itemize}
\end{raggedright}
%
For example, to start a Master thesis in German
one would simply simply place
%
\begin{LaTeXCode}[numbers=none]
\documentclass[master,german]{hgbthesis}
\end{LaTeXCode}
%
at the beginning of the document.



\subsection{Thesis parameters (class \texttt{hgbthesis})}

\texttt{hgbthesis} supports several types of thesis documents.
The following parameters must be specified for \emph{all} types:
%
\begin{itemize}
\item \verb!\title{...}!,
\item \verb!\author{...}!,
\item \verb!\programname{...}!,
\item \verb!\placeofstudy{...}!,
\item \verb!\dateofsubmission{yyyy}{mm}{dd}!.
\end{itemize}
%
\noindent A \emph{Bachelor} thesis requires the following, additional items 
(not relevant for Diploma and Master theses):
%
\begin{itemize}
\item \verb!\thesisnumber{...}!,
\item \verb!\coursetitle{...}!,
\item \verb!\semester{...}!,
\item \verb!\advisor{...}!.
\end{itemize}



\section{Style files and user commands}

The package comes with a set of style (\texttt{*.sty}) files that can be used
independently of the document classes listed above:
\texttt{hgb.sty},
\texttt{hgbabbrev.sty},
\texttt{hgbbib.sty},
\texttt{hgbheadings.sty},
\texttt{hgblistings.sty},
\texttt{hgbmath.sty}.


\subsection{General user commands (\texttt{hgb.sty})}


\begin{itemize}
\item \textbf{\texttt{{\bs}hgbDate}}: Outputs the package version date, 
		\eg, ``\texttt{\hgbDate}''.
\item \textbf{\texttt{{\bs}calibrationbox}}: Inserts a test box for checking the final print size.
\end{itemize}



\subsection{Text commands (\texttt{hgbabbrev.sty})}

\subsubsection*{Special characters:}

\begin{itemize}
\item \textbf{\texttt{{\bs}bs}}: Inserts a backslash character (short for \verb!\textbackslash!).
\item \textbf{\texttt{{\bs}obnh}}: Inserts an optional break with no hyphen (\eg, \verb!PlugIn{\obnh}Filter!).
\end{itemize}



\subsubsection*{German abbreviations:}

\begin{itemize}
\item \textbf{\texttt{{\bs}bzgl}}: bzgl.
\item \textbf{\texttt{{\bs}bzw}}: bzw.
\item \textbf{\texttt{{\bs}ca}}: ca.
\item \textbf{\texttt{{\bs}dah}}: d.\thinspace{}h.
\item \textbf{\texttt{{\bs}Dah}}: D.\thinspace{}h.
\item \textbf{\texttt{{\bs}ds}}: d.\thinspace{}sind
\item \textbf{\texttt{{\bs}etc}}: etc.
\item \textbf{\texttt{{\bs}evtl}}: evtl.
\item \textbf{\texttt{{\bs}ia}}: i.\thinspace{}Allg.
\item \textbf{\texttt{{\bs}sa}}: s.\ auch
\item \textbf{\texttt{{\bs}so}}: s.\ oben
\item \textbf{\texttt{{\bs}su}}: s.\ unten
\item \textbf{\texttt{{\bs}ua}}: u.\thinspace{}a.
\item \textbf{\texttt{{\bs}Ua}}: U.\thinspace{}a.
\item \textbf{\texttt{{\bs}uae}}: u.\thinspace{}\"A.
\item \textbf{\texttt{{\bs}usw}}: usw.
\item \textbf{\texttt{{\bs}uva}}: u.\thinspace{}v.\thinspace{}a.
\item \textbf{\texttt{{\bs}uvm}}: u.\thinspace{}v.\thinspace{}m.
\item \textbf{\texttt{{\bs}va}}: vor allem
\item \textbf{\texttt{{\bs}vgl}}: vgl.
\item \textbf{\texttt{{\bs}zB}}: z.\thinspace{}B.
\item \textbf{\texttt{{\bs}ZB}}: Zum Beispiel
\end{itemize}

\subsubsection*{English abbreviations:}

\begin{itemize}
\item \textbf{\texttt{{\bs}ie}}: \ie
\item \textbf{\texttt{{\bs}eg}}: \eg
\item \textbf{\texttt{{\bs}etc}}: etc.
\item \textbf{\texttt{{\bs}Eg}}: \Eg
\item \textbf{\texttt{{\bs}wrt}}: \wrt
\end{itemize}


\subsection{Bibliography commands (\texttt{hgbbib.sty})}



\begin{itemize}
\item \textbf{\texttt{{\bs}AddBibFile}}: A wrapper to \texttt{biblatex}'s \verb!\addbibresource! macro
(for backward compatibility only).
\item \textbf{\texttt{{\bs}MakeBibliography[\emph{options}]}}: Inserts the reference section or chapter.
By default, references are automatically split into category subsections.%
\footnote{Predefined reference categories are \texttt{literature}, \texttt{avmedia}, \texttt{online} and \texttt{software}.}
Use the option \texttt{nosplit} to produce a traditional (\ie, contiguous) list of references.
\end{itemize}

% \MakeBibliography ... creates a reference section split subsections (default)
% \MakeBibliography[nosplit] ... creates a one-piece reference section


\subsection{Code environments (\texttt{hgblistings.sty})}

The following types of code environments are defined:%
%
\begin{itemize}
\item \textbf{\texttt{CCode}}: for C (ANSI),
\item \textbf{\texttt{CppCode}}: for C++ (ISO),
\item \textbf{\texttt{CsCode}}: for C\#,
\item \textbf{\texttt{CssCode}}: for CSS,
\item \textbf{\texttt{GenericCode}}: for generic code,
\item \textbf{\texttt{HtmlCode}}: for HTML,
\item \textbf{\texttt{JavaCode}}: for Java,
\item \textbf{\texttt{JsCode}}: for JavaScript,
\item \textbf{\texttt{LaTeXCode}}: for \latex,
\item \textbf{\texttt{ObjCCode}}: for ObjectiveC,
\item \textbf{\texttt{PhpCode}}: for PHP,
\item \textbf{\texttt{Swift}}: for Swift,
\item \textbf{\texttt{XmlCode}}: for XML.
\end{itemize}
%
\texttt{hgblistings} is based on the \texttt{listingsutf8}%
\footnote{\url{https://ctan.org/pkg/listingsutf8}}
package, thus any valid \texttt{listings}%
\footnote{\url{https://ctan.org/pkg/listings}}
option may be used; for example, 
the option \texttt{numbers=none} to suppress line numbers:
\begin{LaTeXCode}[numbers=none]
    \begin{JavaCode}[numbers=none]
    ... // Java code comes here
    \end{JavaCode}
\end{LaTeXCode}



\subsection{Mathematical commands (\texttt{hgbmath.sty})}

\texttt{hgbmath} requires (and automatically loads) the \texttt{amsmath}%
\footnote{\url{https://ctan.org/pkg/amsmath}}
package, thus all
commands and symbols of \texttt{amsmath} are available by default.
The following \emph{additional} commands can only be used in math mode:
%
\begin{itemize}
\item \textbf{\texttt{{\bs}Cpx}}: $\Cpx$ (complex numbers),
\item \textbf{\texttt{{\bs}N}}: $\N$ (natural numbers),
\item \textbf{\texttt{{\bs}R}}: $\R$ (real numbers),
\item \textbf{\texttt{{\bs}Q}}: $\Q$ (rational numbers),
\item \textbf{\texttt{{\bs}Z}}: $\Z$ (integer numbers).
\end{itemize}



\subsection{Algorithms (\texttt{hgbalgo.sty})}

\texttt{hgbalgo} is a stand-alone package that is based on -- and extends -- the \texttt{algorithmicx} and 
\texttt{algpseudocode} packages.%
\footnote{\url{https://ctan.org/pkg/algorithmicx}}
It fixes some (mostly indentation-related) problems, adds color and provides some additional
commands. It also loads the \texttt{algorithm}%
\footnote{\url{https://ctan.org/pkg/algorithms}}
package which defines a compatible float container for algorithms:
\verb!\begin{algorithm}! \verb!...! \verb!\end{algorithm}!.


\paragraph{Additional user commands:}
\begin{itemize}
\item 
\textbf{\texttt{{\bs}StateL\{<text>\}}}: Creates a \emph{numbered} statement like \texttt{algorithmicx}'s 
\verb!\State! command but provides consistent indentation on multi-line statements.
Note that the statement \texttt{<text>} must be passed as a single argument in \verb!{...}! brackets.
\item
\textbf{\texttt{{\bs}StateNN[<nesting>]\{<text>\}}}: 
Creates a \emph{non-numbered} statement like \texttt{algorith\-micx}'s \verb!\State! 
command but provides consistent indentation inside nested constructs and over multiple lines.
The optional integer argument \verb!<nesting>! can be used to specify the \emph{nesting depth}
to compensate for a bug in \texttt{algorithmicx} (the nesting level inside a block is not set properly before 
the first \verb!\State!). Omitting the optional argument should give correct indentation in most
cases.
\item
\textbf{\texttt{{\bs}Input\{<text>\}}}:
For describing the input parameters in a procedure's preamble. %Usage: \verb!\Input{<description>}!
\item
\textbf{\texttt{{\bs}Output\{<text>\}}}:
For describing the output values in a procedure's preamble. %Usage: \verb!\Output{<description>}!
\item
\textbf{\texttt{{\bs}Returns\{<text>\}}}:
For describing the return values in a procedure's preamble. %Usage: \verb!\Returns{<description>}!
\end{itemize}

\paragraph{Defined algorithm colors:}
\begin{itemize}
\item[] \textbf{\texttt{AlgKeywordColor}} (for algorithm keywords),
\item[] \textbf{\texttt{AlgProcedureColor}} (for procedure and function names),
\item[] \textbf{\texttt{AlgCommentColor}} (for comments).
\end{itemize}
The above colors can be redefined at any time (see the \texttt{xcolor}%
\footnote{\url{https://ctan.org/pkg/xcolor}} package), \eg, by
\begin{LaTeXCode}[numbers=none]
    \definecolor{AlgKeywordColor}{named}{black}
    \definecolor{AlgProcedureColor}{rgb}{0.0, 0.5, 0.0}     % dark green
\end{LaTeXCode}


\section{Package dependencies}

\begin{sloppypar}
The \texttt{hagenberg-thesis} package builds on the following \latex\ packages:\newline
\texttt{abstract}, 
\texttt{algorithm}, 
\texttt{algorithmicx}, 
\texttt{algpseudocode}, 
\texttt{amsbsy}, 
\texttt{amsfonts}, 
\texttt{amsmath}, 
\texttt{amssymb}, 
\texttt{babel}, 
\texttt{biblatex}, 
\texttt{breakurl}, 
\texttt{caption}, 
\texttt{cmap}, 
\texttt{csquotes}, 
\texttt{datetime}, 
\texttt{enumitem}, 
\texttt{epstopdf}, 
\texttt{eurosym}, 
\texttt{exscale}, 
\texttt{fancyhdr}, 
\texttt{float}, 
\texttt{fontenc}, 
\texttt{geometry}, 
\texttt{graphicx}, 
\texttt{hypcap}, 
\texttt{hyperref}, 
\texttt{ifpdf}, 
\texttt{ifthen}, 
\texttt{inputenc}, 
\texttt{listingsutf8}, 
\texttt{lmodern}, 
\texttt{moreverb}, 
\texttt{overpic}, 
\texttt{pdfpages}, 
\texttt{pict2e}, 
\texttt{subdepth}, 
\texttt{titlesec}, 
\texttt{titling}, 
\texttt{tocloft}, 
\texttt{url}, 
\texttt{upquote}, 
\texttt{verbatim}, 
\texttt{xcolor}, 
\texttt{xifthen}, 
\texttt{xspace}.
\end{sloppypar}


\end{document}
