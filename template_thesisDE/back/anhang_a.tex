\chapter{Technische Informationen}
\label{app:TechnischeInfos}

\newcommand*{\checkbox}{{\fboxsep 1pt%
\framebox[1.30\height]{\vphantom{M}\checkmark}}}

\section{Aktuelle Dateiversionen}

\begin{center}
\begin{tabular}{|l|l|}
\hline
Datum & Datei \\
\hline\hline
\hgbthesisDate & \texttt{hgbthesis.cls} \\
\hline
\hgbDate       & \texttt{hgb.sty} \\
\hline
\end{tabular}
\end{center}




\section{Details zur aktuellen Version}


Das ist eine völlig überarbeitete Version der DA/BA-Vorlage, die
\mbox{UTF-8} kodierten Dateien vorsieht und ausschließlich im PDF-Modus arbeitet.
Der "`klassische"' DVI-PS-PDF-Modus wird somit nicht mehr unterstützt! 

\subsection{Allgemeine technische Voraussetzungen}

Eine aktuelle \latex-Installation mit
\begin{itemize}
	
		\item Texteditor für \mbox{UTF-8} kodierte (Unicode) Dateien,
		\item \texttt{biber}-Programm (BibTeX-Ersatz, Version $\geq 1.5$),
		\item \texttt{biblatex}-Paket (Version $\geq 2.5$, 2013/01/10),
		\item Latin Modern Schriften (Paket \texttt{lmodern}).%
			\footnote{\url{http://www.ctan.org/pkg/lm}, \url{http://www.tug.dk/FontCatalogue/lmodern}}
\end{itemize}


\subsection{Verwendung unter Windows}
\label{sec:VerwendungUnterWindows}

Eine typische Installation unter Windows sieht folgendermaßen aus
(s.\ auch Abschnitt \ref{sec:Windows}):
%
\begin{enumerate}
\item \textbf{MikTeX 2.9}%
	\footnote{\url{http://www.miktex.org/} -- \textbf{Achtung:} 
	Generell wird die \textbf{Komplett\-installation} von MikTeX ("`Complete MiKTeX"') empfohlen, 
	da diese bereits alle notwendigen Zusatzpakete und Schriftdateien enthält! 
	Bei der Installation ist darauf zu achten, 
	dass die automatische Installation erforderlicher Packages 
	durch "`\emph{Install missing packages on-the-fly: = Yes}"' ermöglicht wird (NICHT "`\emph{Ask me first}"')!
	Außerdem ist zu empfehlen, unmittelbar nach der Installation von MikTeX mit dem Programm
	\texttt{MikTeX} $\to$ \texttt{Maintenance} $\to$ \texttt{Update} und \texttt{Package Manager} 
	ein Update der installierten Pakete durchzuführen.}
	(LaTeX-Basisumgebung),
\item \textbf{TeXnicCenter 2.0}%
	\footnote{\url{http://www.texniccenter.org/}}
	(Editor, unterstützt UTF-8),
\item \textbf{SumatraPDF}%
	\footnote{\url{http://blog.kowalczyk.info/software/sumatrapdf/}} 
	(PDF-Viewer).
\end{enumerate}
%
Ein passendes TeXnicCenter-Outputprofil für MikTeX, Biber und Sumatra ist in diesem Paket enthalten.%
\footnote{Datei \nolinkurl{_setup/texniccenter/tc_output_profile_sumatra_utf8.tco}} 
Dieses sollte man zuerst
über \texttt{Build} $\to$ \texttt{Define Output Profiles...} in TeXnicCenter importieren.
\textbf{Achtung}: Alle neu angelegten \texttt{.tex}-Dateien sollten grundsätzlich in UTF-8 Kodierung gespeichert werden!


\subsection{Verwendung unter Mac~OS}

Diese Version sollte insbesondere mit \emph{MacTeX} problemlos laufen (s.\ auch Abschnitt \ref{sec:MacOs}):
\begin{enumerate}
\item 
	\emph{MacTex} (2012 oder höher).
\item 
	Die Zeichenkodierung des Editors sollte auf UTF-8 eingestellt sein.
\item 
	Als Engine (vergleichbar mit den Ausgabeprofilen in TeXnicCenter) sollte \emph{LaTeXMk} verwendet werden. 
	Dieses Perl-Skript erkennt automatisch, wie viele Aufrufe von \emph{pdfLaTeX} und \emph{Biber} nötig sind. 
	Die Ausgabeprofile \emph{LaTeX} oder \emph{pdfLaTeX} hingegen müssen mehrmals aufgerufen werden, 
	zudem werden hierbei auch die Literaturdaten nicht verarbeitet. Dazu müsste extra die \emph{Biber}-Engine 
	aufgerufen werden, 	die jedoch noch nicht in allen Editoren vorhanden ist.
\end{enumerate}
